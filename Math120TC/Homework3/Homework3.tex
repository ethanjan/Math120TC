\documentclass[12pt]{article}
 
\usepackage[margin=1in]{geometry}
\usepackage{amsmath,amsthm,amssymb}
\usepackage{mathtools}
\DeclarePairedDelimiter{\ceil}{\lceil}{\rceil}
%\usepackage{mathptmx}
\usepackage{accents}
\usepackage{comment}
\usepackage{graphicx}
\usepackage{IEEEtrantools}
 \usepackage{float}
 
\newcommand{\N}{\mathbb{N}}
\newcommand{\Z}{\mathbb{Z}}
\newcommand{\R}{\mathbb{R}}
\newcommand{\Q}{\mathbb{Q}}
\newcommand*\conj[1]{\bar{#1}}
\newcommand*\mean[1]{\bar{#1}}
\newcommand\widebar[1]{\mathop{\overline{#1}}}


\newcommand{\cc}{{\mathbb C}}
\newcommand{\rr}{{\mathbb R}}
\newcommand{\qq}{{\mathbb Q}}
\newcommand{\nn}{\mathbb N}
\newcommand{\zz}{\mathbb Z}
\newcommand{\aaa}{{\mathcal A}}
\newcommand{\bbb}{{\mathcal B}}
\newcommand{\rrr}{{\mathcal R}}
\newcommand{\fff}{{\mathcal F}}
\newcommand{\ppp}{{\mathcal P}}
\newcommand{\eps}{\varepsilon}
\newcommand{\vv}{{\mathbf v}}
\newcommand{\ww}{{\mathbf w}}
\newcommand{\xx}{{\mathbf x}}
\newcommand{\ds}{\displaystyle}
\newcommand{\Om}{\Omega}
\newcommand{\dd}{\mathop{}\,\mathrm{d}}
\newcommand{\ud}{\, \mathrm{d}}
\newcommand{\seq}[1]{\left\{#1\right\}_{n=1}^\infty}
\newcommand{\isp}[1]{\quad\text{#1}\quad}
\newcommand*\diff{\mathop{}\!\mathrm{d}}

\DeclareMathOperator{\imag}{Im}
\DeclareMathOperator{\re}{Re}
\DeclareMathOperator{\diam}{diam}
\DeclareMathOperator{\Tr}{Tr}
\DeclareMathOperator{\cis}{cis}

\def\upint{\mathchoice%
    {\mkern13mu\overline{\vphantom{\intop}\mkern7mu}\mkern-20mu}%
    {\mkern7mu\overline{\vphantom{\intop}\mkern7mu}\mkern-14mu}%
    {\mkern7mu\overline{\vphantom{\intop}\mkern7mu}\mkern-14mu}%
    {\mkern7mu\overline{\vphantom{\intop}\mkern7mu}\mkern-14mu}%
  \int}
\def\lowint{\mkern3mu\underline{\vphantom{\intop}\mkern7mu}\mkern-10mu\int}




\newenvironment{theorem}[2][Theorem]{\begin{trivlist}
\item[\hskip \labelsep {\bfseries #1}\hskip \labelsep {\bfseries #2.}]}{\end{trivlist}}
\newenvironment{lemma}[2][Lemma]{\begin{trivlist}
\item[\hskip \labelsep {\bfseries #1}\hskip \labelsep {\bfseries #2.}]}{\end{trivlist}}
\newenvironment{exercise}[2][Exercise]{\begin{trivlist}
\item[\hskip \labelsep {\bfseries #1}\hskip \labelsep {\bfseries #2.}]}{\end{trivlist}}
\newenvironment{problem}[2][Problem]{\begin{trivlist}
\item[\hskip \labelsep {\bfseries #1}\hskip \labelsep {\bfseries #2.}]}{\end{trivlist}}
\newenvironment{question}[2][Question]{\begin{trivlist}
\item[\hskip \labelsep {\bfseries #1}\hskip \labelsep {\bfseries #2.}]}{\end{trivlist}}
\newenvironment{corollary}[2][Corollary]{\begin{trivlist}
\item[\hskip \labelsep {\bfseries #1}\hskip \labelsep {\bfseries #2.}]}{\end{trivlist}}

\newenvironment{solution}{\begin{proof}[Solution]}{\end{proof}}
 
\begin{document}
 
% --------------------------------------------------------------
%                         Start here
% --------------------------------------------------------------
\title{Math 120TC Homework 3}
\author{Ethan Martirosyan}
\date{\today}
\maketitle
\hbadness=99999
\hfuzz=50pt
\section*{Exercise 4.1.1}
\subsection*{Part A}
We define the map $f: \rr^2 \rightarrow X\setminus B^2$ as follows: $f(0) = [0]$ and $f(x) = f(re^{i\theta}) = [(r+1)e^{i\theta}]$ if $x \neq 0$. It is evident that $f$ is bijective. Now, we must show that $f$ and $f^{-1}$ are continuous. First, we claim that $f$ is continuous. Let $U \subseteq X \setminus B^2$ be open. If $U$ does not contain $[0]$, then it is clear that $f^{-1}(U)$ is open. If $[0] \in U$, then we know that $B^2 \subseteq q^{-1}(U)$, where $q: X \rightarrow X\setminus B^2$ is the projection map. Since $U$ is open, $q^{-1}(U)$ is open. Because $B^2$ is closed, we must have  $B^2 \subsetneq q^{-1}(U)$. By compactness, we know that there must be some open ball $D$ of radius strictly greater than $1$ such that $D \subseteq q^{-1}(U)$. Now, we have $U = q(D) \cup U \setminus \{[0]\}$. We note that 
\[
f^{-1}(U) = f^{-1}(q(D)) \cup f^{-1}(U \setminus \{[0]\})
\] which is clearly open since $f^{-1}(q(D))$ and $f^{-1}(U \setminus \{[0]\})$ are open. Next, we claim that $f^{-1}$ is continuous. We note that
\[
f^{-1} \circ q(re^{i\theta}) = (\max\{0,r-1\})e^{i\theta}
\] which is continuous. Thus, if $U \subseteq \rr^2$ is an open set, we have
\[
(f^{-1} \circ q)^{-1}(U) = q^{-1}\circ f(U)
\] is open so that $f(U)$ is open. This shows that $f^{-1}$ is continuous so that $f$ is a homeomorphism.
\newpage
\subsection*{Part B}
Let us suppose that $f: Y \rightarrow X \setminus U$ is a homeomorphism, where $Y$ is any metric space. Since $f$ is bijective, there must exist $y_1$ and $y_2$ in $Y$ such that $f(y_1) = [(0,0)]$ and $f(y_2) = [(1,0)]$. Because $Y$ is a metric space, we have $d(y_1,y_2) > 0$ and let $\delta = (1/2)d(y_1,y_2)$. We consider the open ball $B(y_2; \delta)$. Since this ball is open in $Y$ and $f$ is a homeomorphism, we know that $f(B(y_2;\delta))$ is open so that $q^{-1}(f(B(y_2;\delta)))$ is open. Since $(1,0) \in q^{-1}(f(B(y_2;\delta)))$, there must exist some $\varepsilon < 1$ such that $(\varepsilon, 0) \in q^{-1}(f(B(y_2;\delta)))$ so that $[(0,0)] \in f(B(y_2;\delta))$. That is, there must exist some $y_3 \in B(y_2;\delta)$ such that $f(y_3) = [(0,0)]$. Since $y_3 \neq y_1$, we deduce that $f$ is not injective, which is a contradiction. 
\end{document} 